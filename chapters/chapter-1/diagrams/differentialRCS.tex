% sets up the perspective from which we are looking at the picture {theta}{phi}
% {theta} defines the rotion by x axis while {phi} defines the rotation by z axis
\tdplotsetmaincoords{60}{120}
\begin{figure}
    \centering
    \begin{tikzpicture}
        [scale=4,
        tdplot_main_coords,
        axis/.style={->,black,thick},
        vector/.style={-stealth,very thick},
        vector guide/.style={dashed,red,thick}]

        \coordinate (O) at (0, 0, 0);

        % this should serve as our unit vector
        \def\unit{0.3} 

        \def\ax{0.4}
        \def\ay{0.5}
        \def\az{0.8}

        \def\bx{1}
        \def\by{1}
        \def\bz{1.2}

        \pgfmathsetmacro{\dx}{\bx-\ax}
        \pgfmathsetmacro{\dy}{\by-\ay}
        \pgfmathsetmacro{\dz}{\bz-\az}

        \pgfmathsetmacro{\dxd}{\dx/2}
        \pgfmathsetmacro{\dyd}{\dy/2}
        \pgfmathsetmacro{\dzd}{\dz/2}

        % drawing axes
        \draw[axis] (0, 0, 0) -- (1, 0, 0) node[anchor=north east]{$x$};
        \draw[axis] (0, 0, 0) -- (0, 1, 0) node[anchor=north west]{$y$};
        \draw[axis] (0, 0, 0) -- (0, 0, 1) node[anchor=south]{$z$};

        % drawing the main vector B
        \fill[red] (\ax, \ay, \az) circle[radius=0.2pt] node[right]{};
        \draw[vector, red] (0, 0, 0) -- (\ax, \ay, \az);

        \fill[blue] (\bx, \by, \bz) circle[radius=0.2pt] node[right]{};
        \draw[vector, blue] (0, 0, 0) -- (\bx, \by, \bz);

        \draw (\ax, \ay, \az) -- ++(\dx, 0, 0) -- ++(0, \dy, 0) -- node[right]{$dx$} ++(-\dx, 0, 0) -- ++(0, -\dy, 0);
        \draw (\ax, \ay, \az) -- ++(0, 0, \dz) -- ++(\dx, 0, 0) -- ++(0, \dy, 0) -- ++(-\dx, 0, 0) -- node[above]{$dy$} ++(0, -\dy, 0);
        \draw (\ax+\dx, \ay, \az) -- +(0, 0, \dz);
        \draw (\ax+\dx, \ay+\dy, \az) -- +(0, 0, \dz);
        \draw (\ax, \ay+\dy, \az) -- node[right]{$dz$} +(0, 0, \dz);

        % drawing the projections of red
        \draw[vector guide] (\ax, \ay, \az) -- (\ax, \ay, 0);
        \draw[vector guide] (\ax, 0, 0) -- (\ax, \by, 0);
        \draw[vector guide] (0, \ay, 0) -- (\bx, \ay, 0);

        % drawing the projections of blue
        \draw[vector guide, blue] (\bx, \by, \bz-\dz) -- (\bx, \by, 0);
        \draw[vector guide, blue] (\bx, 0, 0) -- (\bx, \by, 0);
        \draw[vector guide, blue] (0, \by, 0) -- (\bx, \by, 0);

        \draw[vector guide, black] (\ax+\dx, \ay, \az) -- (\ax+\dx, \ay, 0);
        \draw[vector guide, black] (\ax, \ay+\dy, \az) -- (\ax, \ay+\dy, 0);

        % trying to draw shades
        \draw[fill=gray, fill opacity=0.1] (\bx, \by, \bz) -- ++(0, 0, -\dz) -- ++(-\dx, 0, 0) -- ++(0, 0, \dz) -- ++(\dx, 0, 0);
        \draw[vector] (\bx-\dxd, \by, \bz-\dzd) -- ++(0, \dx*\dz, 0) node[right]{$d\vec{S}_y = dxdz\vec{a}_y$};

        \draw[vector] (\bx-\dxd, \by-\dyd, \bz) -- ++(0, 0, \dx*\dy) node[above, xshift=10]{$d\vec{S}_z = dxdy\vec{a}_z$};

        \draw[fill=gray, fill opacity=0.1] (\bx, \by, \bz) -- ++(0, 0, -\dz) -- ++(0, -\dy, 0) -- ++(0, 0, \dz) -- ++(0, \dy, 0);

        \draw[vector] (\bx, \by-\dyd, \bz-\dzd) -- ++(\dy*\dz, 0, 0) node[left, xshift=-11]{$d\vec{S}_x = dydz\vec{a}_x$};

        \draw[fill=gray, fill opacity=0.1] (\ax, \ay, \az) -- ++(0, 0, \dz) -- ++(\dx, 0, 0) -- ++(0, \dy, 0) -- ++(-\dx, 0, 0) -- node[above]{$dy$} ++(0, -\dy, 0);

    \end{tikzpicture}
    \caption{Two closeby points in RCS}
    \label{fig:differentialRCS}
\end{figure}