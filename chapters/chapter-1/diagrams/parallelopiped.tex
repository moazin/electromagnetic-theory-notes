% sets up the perspective from which we are looking at the picture {theta}{phi}
% {theta} defines the rotion by x axis while {phi} defines the rotation by z axis
\tdplotsetmaincoords{60}{45}
\begin{figure}
    \centering
    \begin{tikzpicture}
        [scale=3,
        tdplot_main_coords,
        axis/.style={->,black,thick},
        vector/.style={-stealth,very thick},
        vector guide/.style={dashed,red,thick}]

        \coordinate (O) at (0, 0, 0);
        \coordinate (A) at (1, 0, 0);
        \coordinate (B) at (0, 1, 0);
        \coordinate (C) at (0.5, 0.5, 1);

        \draw[vector, red] (O) -- node[below]{$\vec{A}$}(A);
        \draw[vector, blue] (O) -- node[above]{$\vec{B}$}(B);
        \draw[vector, green] (O) -- node[above]{$\vec{C}$}(C);

        % completing the parallelogram
        \draw (A) -- +(B);
        \draw (B) -- +(A);
        \draw (C) -- +(A);
        \draw (A) -- (1.5, 0.5, 1);
        \draw (B) -- +(C);
        \draw ($ (B) + (A) $) -- +(C);
        \draw ($ (B) + (A) + (C) $) -- ($ (B) + (C) $);
        \draw (C) -- +(B);
        \draw ($ (A) + (C) $) -- +(B);

        \draw[vector] (O) -- node[left]{$\vec{B}\times\vec{C}$} (0, 0, 1);

        \tdplotdefinepoints(0,0,0)(0, 0, 1)(0.5, 0.5, 1);
        \tdplotdrawpolytopearc[<->]{0.3}{above}{$\theta$}
    \end{tikzpicture}
    \caption{The parallelopiped formed by $\vec{A}$, $\vec{B}$ and $\vec{C}$}
    \label{fig:parallelopiped}
\end{figure}