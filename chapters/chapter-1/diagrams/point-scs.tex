\begin{figure}
    \tdplotsetmaincoords{60}{120}
    \centering
    \begin{tikzpicture}
        [scale=4,
        tdplot_main_coords,
        axis/.style={->,black,thick},
        vector/.style={-stealth,very thick},
        vector guide/.style={dashed,red,thick}]

        % defines unit vector length
        \def\unit{0.3}

        % define our point P in spherical coordiantes
        \def\r{0.8}
        \def\phiii{60}
        \def\thetaaa{45}

        % defines the x, y, and z coordinates for point P
        \pgfmathsetmacro{\ax}{\r*sin(\thetaaa)*cos(\phiii)}
        \pgfmathsetmacro{\ay}{\r*sin(\thetaaa)*sin(\phiii)}
        \pgfmathsetmacro{\az}{\r*cos(\thetaaa)}

        % sets the x, y for unit vector a\rho
        \pgfmathsetmacro{\phix}{0.3*cos(\phiii+90)}
        \pgfmathsetmacro{\phiy}{0.3*sin(\phiii+90)}

        % sets the x, y, z for unit vector a_r
        \pgfmathsetmacro{\unitx}{\unit*sin(\thetaaa)*cos(\phiii)}
        \pgfmathsetmacro{\unity}{\unit*sin(\thetaaa)*sin(\phiii)}
        \pgfmathsetmacro{\unitz}{\unit*cos(\thetaaa)}

        % sets the x, y, z for unit vector a_\Theta
        \pgfmathsetmacro{\thetaz}{\unit*cos(\thetaaa+90)}
        \pgfmathsetmacro{\thetay}{\unit*sin(\thetaaa+90)*sin(\phiii)}
        \pgfmathsetmacro{\thetax}{\unit*sin(\thetaaa+90)*sin(\phiii)}

        \coordinate (O) at (0, 0, 0); % denote origin by (O)
        \coordinate (P) at (\ax, \ay, \az); % denote point P by (P)

        % these should serve as our axes
        \draw[axis] (0, 0, 0) -- (1, 0, 0) node[anchor=north east]{$x$};
        \draw[axis] (0, 0, 0) -- (0, 1, 0) node[anchor=north west]{$y$};
        \draw[axis] (0, 0, 0) -- (0, 0, 1) node[anchor=south]{$z$};

        \draw[blue] (O) -- node[right]{$\rho$} (\ax, \ay, 0); % draws rho line
        % \draw[dashed] (\rholength, 0, 0) arc (0:90:\rholength); % draws the circle r=rho on xy plane
        \draw[red, dashed] (P) -- (\ax, \ay, 0); % line between point P and its xy plane projection

        \draw[black] (O) -- (P); % drawing the r part

        \fill[red] (P) circle[radius=0.5pt] node[right]{$P(r, \phi, \theta)$}; % point P as a filled circle

        % drawing the arc of the angle between rho and x axis
        \tdplotdrawarc[magenta]{(0,0,0)}{0.2}{0}{\phiii}{below}{$\phi$}

        % create a theta plane basically
        \tdplotsetthetaplanecoords{\phiii}

        %draw theta arc and label, using rotated coordinate system
        \tdplotdrawarc[tdplot_rotated_coords, cyan]{(0,0,0)}{0.2}{0}{\thetaaa}{anchor=south west}{$\theta$}

        % drawing the three unit vectors
        \draw[vector, red] (O) -- node[right, yshift=-3]{$\vec{a}_r$} (\unitx, \unity, \unitz);
        \draw[vector, magenta] (\ax, \ay, 0) -- node[below]{$\vec{a}_\phi$} +(\phix, \phiy, 0); 
        \draw[vector, cyan] (\ax, \ay, \az) -- node[right]{$\vec{a}_\theta$} +(\thetax, \thetay, \thetaz);

    \end{tikzpicture}
    \caption{A point in Spherical Coordinate Systems}
    \label{fig:point-scs}
\end{figure}