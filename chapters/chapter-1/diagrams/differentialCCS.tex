% sets up the perspective from which we are looking at the picture {theta}{phi}
% {theta} defines the rotion by x axis while {phi} defines the rotation by z axis
\tdplotsetmaincoords{60}{10}
\begin{figure}
    \centering
    \begin{tikzpicture}
        [scale=4,
        tdplot_main_coords,
        axis/.style={->,black,thick},
        vector/.style={-stealth,very thick},
        vector guide/.style={dashed,red,thick}]

        \coordinate (O) at (0, 0, 0);

        % this should serve as our unit vector
        \def\unit{0.3} 

        \def\ax{0.6}
        \def\ay{0.4}
        \def\az{0.8}

        \def\bx{0.7}
        \def\by{0.7}
        \def\bz{1.2}

        \pgfmathsetmacro{\dx}{\bx-\ax}
        \pgfmathsetmacro{\dy}{\by-\ay}
        \pgfmathsetmacro{\dz}{\bz-\az}

        \pgfmathsetmacro{\dxd}{\dx/2}
        \pgfmathsetmacro{\dyd}{\dy/2}
        \pgfmathsetmacro{\dzd}{\dz/2}

        \pgfmathsetmacro{\rhored}{sqrt(\ax^2+\ay^2)}
        \pgfmathsetmacro{\rhoblue}{sqrt(\bx^2+\by^2)}

        \pgfmathsetmacro{\drho}{sqrt(\dxd^2+\dyd^2)}

        % drawing axes
        \draw[axis] (0, 0, 0) -- (1, 0, 0) node[anchor=north east]{$x$};
        \draw[axis] (0, 0, 0) -- (0, 1, 0) node[anchor=north west]{$y$};
        \draw[axis] (0, 0, 0) -- (0, 0, 1.5) node[anchor=south]{$z$};

        % drawing the main vector B
        \fill[red] (\ax, \ay, \az) circle[radius=0.2pt] node[right]{};
        \draw[vector, red] (0, 0, 0) -- (\ax, \ay, \az);

        \fill[blue] (\bx, \by, \bz) circle[radius=0.2pt] node[right]{};
        \draw[vector, blue] (0, 0, 0) -- (\bx, \by, \bz);

        % draw their projections on the xy plane
        \draw[dashed, blue] (\bx, \by, \bz) -- (\bx, \by, 0);
        \draw[dashed, red] (\ax, \ay, \az) -- (\ax, \ay, 0);

        % draw rho circles
        \tdplotdrawarc[dashed, red]{(0, 0, 0)}{\rhored}{0}{90}{}{};
        \tdplotdrawarc[dashed, blue]{(0, 0, 0)}{\rhoblue}{0}{90}{}{};

        % radial lines
        \pgfmathsetmacro{\lengthratio}{(sqrt(\bx^2+\by^2))/(sqrt(\ax^2+\ay^2))}
        \pgfmathsetmacro{\lengthratioinverse}{(sqrt(\ax^2+\ay^2))/(sqrt(\bx^2+\by^2))}

        \draw[dashed, red] (O) -- (\ax*\lengthratio, \ay*\lengthratio, 0);
        \draw[dashed, blue] (O) -- (\bx, \by, 0);

        % let's define our cyldrindical vars
        \pgfmathsetmacro{\phiii}{atan(\ay/\ax)}
        \pgfmathsetmacro{\phiiis}{atan(\by/\bx)}
        \tdplotdrawarc[red]{(0, 0, \az)}{\rhored}{\phiii}{\phiiis}{}{};
        \tdplotdrawarc[red]{(0, 0, \bz)}{\rhored}{\phiii}{\phiiis}{}{};
        \tdplotdrawarc[blue]{(0, 0, \az)}{\rhoblue}{\phiii}{\phiiis}{}{};
        \tdplotdrawarc[blue]{(0, 0, \bz)}{\rhoblue}{\phiii}{\phiiis}{}{};
        % draw rho lines
        \draw (0, 0, \az) -- (\ax*\lengthratio , \ay*\lengthratio, \az);
        \draw (0, 0, \bz) -- (\ax*\lengthratio , \ay*\lengthratio, \bz);
        \draw (\ax, \ay, \az) -- (\ax, \ay, \bz);
        \draw (\ax*\lengthratio, \ay*\lengthratio, \az) -- +(0 ,0, \dz);

        \draw (0, 0, \az) -- (\bx , \by, \az);
        \draw (0, 0, \bz) -- (\bx , \by, \bz);
        \draw (\bx*\lengthratioinverse, \by*\lengthratioinverse, \az) -- +(0, 0, \dz);
        \draw (\bx, \by, \az) -- (\bx, \by, \bz);

        \draw[fill=gray, fill opacity=0.1] (\ax, \ay, \az) -- (\ax*\lengthratio, \ay*\lengthratio, \az) -- +(0, 0, \dz) -- (\ax, \ay, \bz) -- +(0, 0, -\dz);
        \draw[fill=gray, fill opacity=0.1] (\bx, \by, \bz) -- +(0, 0, -\dz) -- (\bx*\lengthratioinverse, \by*\lengthratioinverse, \az) -- +(0, 0, \dz);
        \draw[fill=gray, fill opacity=0.1] (\ax, \ay, \az) -- +(0, 0, \dz) -- (\bx*\lengthratioinverse, \by*\lengthratioinverse, \bz) -- +(0, 0, -\dz);
        \draw[fill=gray, fill opacity=0.1] (\bx, \by, \bz) -- (\bx*\lengthratioinverse, \by*\lengthratioinverse, \bz) -- (\ax, \ay, \bz) -- (\ax*\lengthratio, \ay*\lengthratio, \bz) -- (\bx, \by, \bz);

    \end{tikzpicture}
    \caption{Two closeby points in CCS}
    \label{fig:differentialCCS}
\end{figure}