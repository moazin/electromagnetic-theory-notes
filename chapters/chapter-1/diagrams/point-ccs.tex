% sets up the perspective from which we are looking at the picture {theta}{phi}
% {theta} defines the rotion by x axis while {phi} defines the rotation by z axis
\begin{figure}
    \tdplotsetmaincoords{60}{120}
    \centering
    \begin{tikzpicture}
        [scale=4,
        tdplot_main_coords,
        axis/.style={->,black,thick},
        vector/.style={-stealth,very thick},
        vector guide/.style={dashed,red,thick}]
        
        % defines unit vector length
        \def\unit{0.3}

        % defines the x, y, and z coordinates for point P
        \def\ax{0.5}
        \def\ay{0.5}
        \def\az{0.5}

        \coordinate (O) at (0, 0, 0); % denote origin by (O)
        \coordinate (P) at (\ax, \ay, \az); % denote point P by (P)

        \def\rholength{ {sqrt(\ax^2+\ay^2)} } % \rholength stores the distance between Pxy and O

        % these should serve as our axes
        \draw[axis] (0, 0, 0) -- (1, 0, 0) node[anchor=north east]{$x$};
        \draw[axis] (0, 0, 0) -- (0, 1, 0) node[anchor=north west]{$y$};
        \draw[axis] (0, 0, 0) -- (0, 0, 1) node[anchor=south]{$z$};

        \draw[blue] (O) -- node[right]{$\rho$} (\ax, \ay, 0); % draws rho line
        \draw[dashed] (\rholength, 0, 0) arc (0:90:\rholength); % draws the circle r=rho on xy plane
        \draw[red] (P) -- node[right]{$z$} (\ay, \ay, 0); % line between point P and its xy plane projection

        \fill[red] (P) circle[radius=0.5pt] node[right]{$P(\rho, \phi, z)$}; % point P as a filled circle

        % drawing the arc of the angle between rho and x axis
        \tdplotdefinepoints(0,0,0)(1, 0, 0)(\ax, \ax, 0); 
        \tdplotdrawpolytopearc[-]{0.2}{below}{$\phi$}

        % setting up some lengths for drawing unit vectors
        \def\unitvectorrhox{ {\unit*cos{45}} }
        \def\unitvectorrhoy{ {\unit*sin{45}} }

        \def\unitvectorphix{ {\unit*cos{135}} }
        \def\unitvectorphiy{ {\unit*sin{135}} }

        \draw[vector, red] (O) -- (0, 0, \unit) node[right]{$\vec{a}_z$}; % unit vector az
        \draw[vector, blue] (O) -- node[right, yshift=10]{$\vec{a}_\rho$} (\unitvectorrhox, \unitvectorrhoy, 0); % unit vector a\rho
        \draw[vector] (0.5, 0.5, 0) -- node[below]{$\vec{a}_\phi$} +(\unitvectorphix, \unitvectorphiy, 0); % unit vector a\phi

    \end{tikzpicture}
    \caption{A point $P$ represented in CCS}
    \label{fig:pointinccs}
\end{figure}